%%%%%%%%%%%%%%%%%%%%%%%%%%%%%%%%%%%%%%%%%%%%%%%%%%%%%%%%%%%%%%%%%%%%%%%%%%%%%%%%%%%%%%
%%%%%%%%%%%%%%%%%%%%%%%%%%%%%%%%%% Settings %%%%%%%%%%%%%%%%%%%%%%%%%%%%%%%%%%%%%%%%%%
%%%%%%%%%%%%%%%%%%%%%%%%%%%%%%%%%%%%%%%%%%%%%%%%%%%%%%%%%%%%%%%%%%%%%%%%%%%%%%%%%%%%%%

\documentclass{beamer}
\usepackage[utf8]{inputenc}
\usepackage[english]{babel}
\usepackage[T1]{fontenc}
\usepackage{appendixnumberbeamer}
\usepackage{siunitx}

\usepackage{minted}
\setminted{linenos, numberblanklines=false, mathescape, texcomments, autogobble, breakanywhere, breakautoindent, breaklines, frame=lines, framesep=5\fboxsep}
\setminted[python]{python3}

\usefonttheme{professionalfonts}
\usepackage{mathspec}
\setsansfont[BoldFont={Fira Sans},
Numbers={OldStyle}]{Fira Sans Light}
\setmathsfont(Digits)[Numbers={Lining, Proportional}]{Fira Sans Light}

\def\mytitle{Production Line Performance}
\def\mysubtitle{Project work in Languages and Algorithms for Artificial Intelligence class (Module 2)}
\def\myshortfullname{A. Falai}
\def\myfullname{Alessio Falai}
\def\myemail{alessio.falai@studio.unibo.it}
\def\myshortinstitute{UNIBO}
\def\myinstitute{Alma Mater Studiorum - University of Bologna}
\title{\mytitle}
\subtitle{\mysubtitle}
\author[\myshortfullname]{\myfullname \newline \texttt{\myemail}}
\date{\today}
\institute[\myshortinstitute]{\myinstitute}
\titlegraphic{\hfill\includegraphics[height=1.5cm]{img/unibo-logo.eps}}

\usetheme[progressbar=frametitle, numbering=none, background=light, titleformat=smallcaps, block=fill]{metropolis}
\useoutertheme{metropolis}
\useinnertheme{metropolis}
\usefonttheme{metropolis}
\usecolortheme{metropolis}
\setbeamercovered{dynamic}

% Edit progressbar width
\makeatletter
\setlength{\metropolis@progressinheadfoot@linewidth}{1.5pt}
\setlength{\metropolis@progressonsectionpage@linewidth}{1.5pt}
\makeatother

% Show 'page' before page numbers in toc
\addtocontents{toc}{~\hfill page\par}

% Show page numbers and dots in toc
\makeatletter
\long\def\beamer@section[#1]#2{%
  \beamer@savemode%
  \mode<all>%
  \ifbeamer@inlecture
    \refstepcounter{section}%
    \beamer@ifempty{#2}%
    {\long\def\secname{#1}\long\def\lastsection{#1}}%
    {\global\advance\beamer@tocsectionnumber by 1\relax%
      \long\def\secname{#2}%
      \long\def\lastsection{#1}%
      \addtocontents{toc}{\protect\beamer@sectionintoc{\the\c@section}{#2\dotfill\the\c@page}{\the\c@page}{\the\c@part}%
        {\the\beamer@tocsectionnumber}}}%
    {\let\\=\relax\xdef\sectionlink{{Navigation\the\c@page}{\noexpand\secname}}}%
    \beamer@tempcount=\c@page\advance\beamer@tempcount by -1%
    \beamer@ifempty{#1}{}{%
      \addtocontents{nav}{\protect\headcommand{\protect\sectionentry{\the\c@section}{#1}{\the\c@page}{\secname}{\the\c@part}}}%
      \addtocontents{nav}{\protect\headcommand{\protect\beamer@sectionpages{\the\beamer@sectionstartpage}{\the\beamer@tempcount}}}%
      \addtocontents{nav}{\protect\headcommand{\protect\beamer@subsectionpages{\the\beamer@subsectionstartpage}{\the\beamer@tempcount}}}%
    }%
    \beamer@sectionstartpage=\c@page%
    \beamer@subsectionstartpage=\c@page%
    \def\insertsection{\expandafter\hyperlink\sectionlink}%
    \def\insertsubsection{}%
    \def\insertsubsubsection{}%
    \def\insertsectionhead{\hyperlink{Navigation\the\c@page}{#1}}%
    \def\insertsubsectionhead{}%
    \def\insertsubsubsectionhead{}%
    \def\lastsubsection{}%
    \Hy@writebookmark{\the\c@section}{\secname}{Outline\the\c@part.\the\c@section}{2}{toc}%
    \hyper@anchorstart{Outline\the\c@part.\the\c@section}\hyper@anchorend%
    \beamer@ifempty{#2}{\beamer@atbeginsections}{\beamer@atbeginsection}%
  \fi%
  \beamer@resumemode}%
\makeatother

% Change shape size for itemize
\setbeamertemplate{itemize item}[circle]
\setbeamertemplate{itemize subitem}[square]

% Conversion to roman numbers
\makeatletter
\newcommand*{\rom}[1]{\expandafter\@slowromancap\romannumeral #1@}
\makeatother

%%%%%%%%%%%%%%%%%%%%%%%%%%%%%%%%%%%%%%%%%%%%%%%%%%%%%%%%%%%%%%%%%%%%%%%%%%%%%%%%%%%%%%
%%%%%%%%%%%%%%%%%%%%%%%%%%%%%%%%%% Document %%%%%%%%%%%%%%%%%%%%%%%%%%%%%%%%%%%%%%%%%%
%%%%%%%%%%%%%%%%%%%%%%%%%%%%%%%%%%%%%%%%%%%%%%%%%%%%%%%%%%%%%%%%%%%%%%%%%%%%%%%%%%%%%% 

\begin{document}

%%%%%%%%%%%%%%%%%%%%%%%%%%%%%%%%%%%%%%%%%%%%%%%%%%%%%%%%%%%%%%%%%%%%%%%%%%%%%%%%%%%%%%
%%%%%%%%%%%%%%%%%%%%%%%%%%%%%%%%% Title & TOC %%%%%%%%%%%%%%%%%%%%%%%%%%%%%%%%%%%%%%%%
%%%%%%%%%%%%%%%%%%%%%%%%%%%%%%%%%%%%%%%%%%%%%%%%%%%%%%%%%%%%%%%%%%%%%%%%%%%%%%%%%%%%%% 

\renewcommand*{\insertpagenumber}{%
	\Roman{framenumber}%
}%
\makeatother
\setbeamertemplate{footline}
{
  \leavevmode%
  \hbox{%
  \begin{beamercolorbox}[wd=.3\paperwidth,ht=2.25ex,dp=1ex,center]{author in head/foot}%
    \usebeamerfont{author in head/foot}\insertshortauthor ~ (\myshortinstitute)
  \end{beamercolorbox}%
  \begin{beamercolorbox}[wd=.6\paperwidth,ht=2.25ex,dp=1ex,center]{block title}%
    \usebeamerfont{title in head/foot}\insertshorttitle
  \end{beamercolorbox}%
  \begin{beamercolorbox}[wd=.1\paperwidth,ht=2.25ex,dp=1ex,center]{block body}%
    \insertpagenumber{} \hspace*{1ex}
  \end{beamercolorbox}}%
  \vskip0pt%
}
\makeatletter
\setbeamertemplate{navigation symbols}{}

\begin{frame}
	\maketitle
\end{frame}

\begin{frame}[allowframebreaks]
  \frametitle{Table of contents}
  \setbeamertemplate{section in toc}[ball]
  \tableofcontents[sections={1,5,6}, hideallsubsections]
    \framebreak
  \tableofcontents[sections={2,3,4}, hideallsubsections]
\end{frame}

%%%%%%%%%%%%%%%%%%%%%%%%%%%%%%%%%%%%%%%%%%%%%%%%%%%%%%%%%%%%%%%%%%%%%%%%%%%%%%%%%%%%%%
%%%%%%%%%%%%%%%%%%%%%%%%%%%%%%%%% Main content %%%%%%%%%%%%%%%%%%%%%%%%%%%%%%%%%%%%%%%
%%%%%%%%%%%%%%%%%%%%%%%%%%%%%%%%%%%%%%%%%%%%%%%%%%%%%%%%%%%%%%%%%%%%%%%%%%%%%%%%%%%%%% 

\setcounter{framenumber}{0}
\makeatother
\setbeamertemplate{footline}
{
  \leavevmode%
  \hbox{%
  \begin{beamercolorbox}[wd=.3\paperwidth,ht=2.25ex,dp=1ex,center]{author in head/foot}%
    \usebeamerfont{author in head/foot}\insertshortauthor ~ (\myshortinstitute)
  \end{beamercolorbox}%
  \begin{beamercolorbox}[wd=.6\paperwidth,ht=2.25ex,dp=1ex,center]{block title}%
    \usebeamerfont{title in head/foot}\insertshorttitle
  \end{beamercolorbox}%
  \begin{beamercolorbox}[wd=.1\paperwidth,ht=2.25ex,dp=1ex,center]{block body}%
    \insertframenumber{} / \inserttotalframenumber\hspace*{1ex}
  \end{beamercolorbox}}%
  \vskip0pt%
}
\makeatletter
\setbeamertemplate{navigation symbols}{}

% Datasets section
\section{Datasets}
\begin{frame}
  \frametitle{Test datasets}
  \begin{itemize}[<+->]
    \item \textbf{Arrest}: Contains statistics, in arrests per 100.000 residents, for assault, murder, and rape in each of the 50 US states in 1973. 
          It also gives the percent of the population living in urban areas.
    \item \textbf{Adult}: Aims at separating people whose income is greater than 50 thousands dollars per year from the rest.
  \end{itemize}	
\end{frame}

\begin{frame}
  \frametitle{Actual dataset}
  \begin{itemize}[<+->]
    \item \textbf{Bosch}: Aims at predicting internal failures using thousands of measurements and tests made for each component along different assembly lines.
  \end{itemize}
\end{frame}

% Preprocessing section
\section{Preprocessing}
\begin{frame}
  \frametitle{Data exploration}
  \begin{itemize}[<+->]
    \item The Bosch dataset includes 3 subsets: numerical, categorical and time data.
    \item As stated in \cite{predict-failures}, the categorical data is extremely sparse, and thus not exploited in subsequent stages.
    \item Our first analysis will be focused on just \textbf{numerical data}:
    \begin{itemize} 
      \item $968$ anonymized features
      \item<.-> \num{1183747} labeled examples
      \item<.-> $0.58\%$ of failed products
      \item<.-> $78.5\%$ of missing values
    \end{itemize}
  \end{itemize}
\end{frame}

\begin{frame}
  \frametitle{Two-stage approach}
  \begin{itemize}[<+->]
    \item \textbf{Stage \rom{1}}: This step clusters data with similar processes together into process groups.
    \item \textbf{Stage \rom{2}}: This step uses supervised learning to predict the failed products. Each cluster is treated as an independent dataset and has its own classifier.
  \end{itemize}
\end{frame}

\begin{frame}
  \frametitle{Data preprocessing}
  \begin{itemize}[<+->]
    \item \textbf{Common}: Columns containing null values and constant values in each row are dropped.
    \item \textbf{Clustering}: Values are binarized (0 meaning null value and 1 meaning not null value) and \textit{PCA} is applied to binarized features.
    \item \textbf{Classification}: A \textit{feature imputation} method (mean value over the column), followed by \textit{feature standardization} (zero mean, unit variance) and \textit{PCA}, is applied over non-binary values in each cluster.
    \item \textbf{Prediction}: A new example follows the same preprocessing scheme as the whole dataset.
  \end{itemize}
\end{frame}

\begin{frame}
  \frametitle{PCA}
  Custom implementation of the \textit{PCA} workflow, taking into account:
  \begin{itemize}[<+->]
    \item \textit{Features assembly} and \textit{features standardization}.
    \item Selection of the minimum number of principal components \textit{explaining} the given percentage of \textit{variance} in the data (defaults to $95\%$).
    \item \textit{Transformation matrix} storage, so that new examples can be easily converted into principal components.
  \end{itemize}
\end{frame}

% Clustering section
\section{Clustering}
\begin{frame}
  \frametitle{k-means}
    The chosen clustering algorithm is \texttt{k-means}, since density-based methods (like \texttt{DBSCAN}) are still not available in Spark. So, the following issues needed to be addressed:
    \begin{itemize}[<+->]
      \item \textbf{Problem}: Automatically select the right amount of clusters:
      \begin{itemize}
        \item \textit{Silhouette} and \textit{elbow} methods are not ideal, since they require human analysis.
        \item<.-> Spark 3.0.0 dropped support for \textit{inertia} computation, maintaining only evaluation by silhouette scores.
      \end{itemize} 
      \item \textbf{Solution}: Ad-hoc implementation of the \textit{Gap statistic} method, described in \cite{gap}.
    \end{itemize}
\end{frame}

\begin{frame}
  \frametitle{Gap statistic}
    \begin{itemize}[<+->]
      \item For $k = 1, .., K$, where $K$ is the maximum number of clusters, perform k-means and compute the resulting inertia $I_{k}$.
      \item Generate $B$ reference datasets by sampling from a uniform distribution over each feature, where the support is directly identified by features ranges. Then, for $k = 1, .., K$ and $b = 1, .., B$, perform k-means and compute the resulting inertia $I_{kb}$. Finally, estimate $E^{*}[log(I_{kb})]$ as $\frac{1}{B} \sum_{b}log(I_{kb})$.
      \item Compute the Gap score as $Gap(k) = E^{*}[log(I_{kb})] - log(I_{k})$.
      \item Compute the standard deviation $sd_{k}$ of $log(I_{kb})$ and define $s_{k} = sd_{k} * \sqrt{1 + \frac{1}{B}}$.
      \item Select the minimum $k$ s.t. $Gap(k) - Gap(k + 1) + s_{k + 1} \geq 0$.
    \end{itemize}
\end{frame}

% Classification section
\section{Classification}
\begin{frame}[<+->]
  \frametitle{Classifiers}
    \begin{itemize}
      \item \textbf{Implemented models}: Decision Tree, \textbf{Random Forest} and Gradient Boosted Tree.
      \item \textbf{Training strategy}: Hyper-parameters selected by cross-validation on a parameter grid, mainly consisting of the following:
      \begin{itemize}
        \item Maximum \textit{depth} of each tree, ranging from 1 to 30 with step 5.
        \item<.-> Minimum \textit{number of instances} each child must have after split, ranging from 1 to 10 with step 2. 
      \end{itemize}
    \end{itemize}
\end{frame}

\begin{frame}
  \frametitle{Evaluation}
  Different evaluation strategies, based on a custom \textit{confusion matrix} computation:
    \begin{itemize}[<+->]
      \item \textbf{Accuracy}: $\frac{TP + TN}{TP + FP + FN + TN}$ (not well-suited to the Bosch dataset, given its high class imbalance).
      \item \textbf{$F_{1}$-score}: $2\times\frac{PPV \times TPR}{PPV + TPR}$, where $PPV = \frac{TP}{TP + FP}$ and $TPR = \frac{TP}{TP + FN}$.
      \item \textbf{Area under ROC}: The two-dimensional area underneath the entire ROC curve ($TPR$/$FPR$ at varying thresholds) from $(0,0)$ to $(1,1)$, where $FPR = \frac{FP}{FP + TN}$.
      \item \textbf{Matthew's Correlation Coefficient (MCC)}: $\frac{TP \times TN - FP \times FN}{\sqrt{(TP + FP) \times (TP + FN) \times (TN + FP) \times (TN + FN)}}$. 
    \end{itemize}
\end{frame}

% Cloud section
\section{Execution}
\begin{frame}
  \frametitle{Cloud issues}
  \begin{itemize}[<+->]
    \item \texttt{Spark 3.0.0} is not yet available on the AWS \texttt{EMR} service (the last version that can be used is \texttt{2.4.4}).
    \item Spark 3.0.0 could be manually installed to \texttt{EC2} clusters, but the \texttt{Flintrock} service already provides some shortcuts to create the desidered clusters and automatically install the selected Spark/Hadoop version.
    \item Unfortunately, Flintrock was not tested with Spark 3.0.0, which in some installations relies on \texttt{Java 11} (as opposed to Spark 2.4.4 which simply uses \texttt{Java 8}).
  \end{itemize}
\end{frame}

\begin{frame}
  \frametitle{Machines}
  \begin{itemize}[<+->]
    \item \textbf{Local}:
    \begin{itemize}
      \item<.-> \textbf{Type}: Macbook Pro 16-inch 2019
      \item<.-> \textbf{CPU}: 2.3 Ghz 8-Core Intel Core i9
      \item<.-> \textbf{RAM}: 16 GB 2667 MHz DDR4
    \end{itemize}
    \item \textbf{Cloud}: 
      \begin{itemize}
        \item<.-> \textbf{Type}: \texttt{t2.xlarge}
        \item<.-> \textbf{CPU}: 4 vCPUs based on Intel Xeon with Intel AVX, Intel Turbo
        \item<.-> \textbf{RAM}: 16 GB
    \end{itemize}
  \end{itemize}
\end{frame}

% Results section
\section{Results}

%%%%%%%%%%%%%%%%%%%%%%%%%%%%%%%%%%%%%%%%%%%%%%%%%%%%%%%%%%%%%%%%%%%%%%%%%%%%%%%%%%%%%%
%%%%%%%%%%%%%%%%%%%%%%%%%%%%%%%%%%% Appendix %%%%%%%%%%%%%%%%%%%%%%%%%%%%%%%%%%%%%%%%%
%%%%%%%%%%%%%%%%%%%%%%%%%%%%%%%%%%%%%%%%%%%%%%%%%%%%%%%%%%%%%%%%%%%%%%%%%%%%%%%%%%%%%% 

\appendix

\begin{frame}[standout]
	Thank you for your attention
\end{frame}

\begin{frame}
	\frametitle{References}
	\nocite{*}
    \bibliography{bibliography}
    \bibliographystyle{plain}
\end{frame}

\end{document}
